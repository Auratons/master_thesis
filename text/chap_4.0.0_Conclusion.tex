\chapter*{Conclusion}
\addcontentsline{toc}{chapter}{Conclusion}

This thesis examines the usage of various rendering techniques in the InLoc
algorithm solving the visual localization problem and its verification step.
The point cloud rendering approaches, and later localization performance are
evaluated on three different dataset types covering both exteriors and
interiors; the InLoc implementation is generalized so that a general dataset,
not only the one that InLoc was released with, can be inputted for localization.

Four different rendering approaches are utilized---as a baseline approach,
the default OpenGL point rendering primitive \verb|GL_POINTS| is used, further,
point splatting and ray marching with signed distance fields are explored.
Finally, aside from the three classical rendering approaches, a neural
rendering deep neural network model is compared with the previous ones. As not
all mentioned renderers were in existence prior to the thesis with sufficient
performance to be able to render point clouds of sizes present in the datasets,
third-party point splatting C++ implementation within a graphical interface is
enhanced with headless rendering capabilities, the capability to read external point
clouds and camera parameters, and output depth information for renders. The ray
marching renderer is implemented in C++ and CUDA from the ground, eventually
reusing the same components from the splatter enhancements. To the best of our knowledge,
previously, there were no such implementations with these capabilities being able
to render tens of millions of points in a reasonable time.\\

We considered the renderers from various angles---rendering performance from visual
and statistical perspective, from computational performance, and finally, from
influence on the localization performance. We show that aside from computational
perspective, the four renderers split roughly into three groups: the predominantly
least performant algorithm is Pyrender, followed by the Splatter and Marcher
implementations with the NRIW model on the top.

\emph{Pyrender} suffers from its slower
implementation in Python and from the primitive it uses as the points have fixed
size in the screen space, which defies perspective drawing principles, leading to
visual artifacts in the form of occlusion problems.

\emph{Splatter} and \emph{Marcher} are less easy to separate. Their principle
is similar as it uses diameters assigned to points and renders them as splats or
spheres. In practical use cases, there are differences, however.
The Splatter requires a normal vector per point on top of diameters to
properly put a face to the splats. In the case of all datasets explored in the
thesis, we did not encounter a situation where a dataset would simultaneously
explore one space from the inside and outside. However, if this happens, Splatter
would require additional functionality to dynamically compute normals per view
or switch directions based on some condition fulfillment. On the other hand,
Splatter shows less dependency on the view frustum contents, whereas Marcher
performs much less consistently. For some views, the rendering time may be
considerably longer than for others. There is room for improvements in the
implementation that may mitigate this issue, including the possible memory caching
boundary hit, causing extremely prolonged rendering times in some cases. Visually,
Splatter can represent corners and edges with less blur. Finally, both
renderers increase the performance of the InLoc pipeline when used for the
dataset's transformation into the InLoc format and for training a neural model
used in the verification step.

The \emph{NRIW} model further pushes localization performance due to its more
realistically-looking rendering capabilities that get exploited in the verification
step of the localization pipeline. The disadvantages and the price for the localization
precision gains are speed, as the model needs, on top of its own slower runtime,
a proxy render of a point cloud from a candidate position generated by another
non-neural renderer, and also its scene dependency that requires training for
every dataset explored. There has been considerable progress in the neural rendering
field since work on the thesis started, so future work may explore these advancements.\\

To summarize, when maximal localization performance is sought, Splatter and Marcher
help the localization pipeline's frontend, together with neural rendering based on the
same. For concrete use cases, other differentiating factors can help to choose a specific
renderer. When the time of answering a localization query is to be minimized, it may be
worth sacrificing some precision by either using the non-neural renderers for the whole
pipeline or, as we show, by lowering the number of points in the scene model that is from
both statistical and visual point of view comparable with the advantage of faster
rendering times. Future work may analyze the effect of lowering the resolution of the
database and query images to inspect the computational performance further.

