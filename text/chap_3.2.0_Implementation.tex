\section{Implementation}

For purposes of the thesis, several code projects are leveraged, either built from the
ground up by the author or based on top of the previous work of others.  Localization
InLoc framework is based on the work of the article author Hajime Taira and further
enhancements done by Pavel Lucivnak and Bastien Dechamps spread across several code
repositories. From renderers, for Neural Rerendering in the Wild, the authors'
implementation with surrounding scripts written by Bastien Dechamps is used as the base of
further work. For surface splatting, the great work of Sebastian Lipponer, with some
tweaks, is leveraged. Finally, the ray marching renderer based on OpenGL is entirely the
author's work.

Aside from the localization algorithm and renderers, scripts transforming dataset formats,
described in~\cref{tab:model_conventions}, into notations and conventions used by InLoc
and renderers described in~\cref{tab:agorithm_conventions} are also added; for further
information, please see below.\\

To be runnable in CIIRC computational cluster environment, which distributes jobs
submitted by users by Slurm\footnote{\url{https://www.schedmd.com}}, batch job shell
scripts are written. Slurm is an open-source, fault-tolerant, and highly scalable cluster
management and job scheduling system for clusters of Linux-running machines. Slurm
requires the batch job scripts to specify memory, CPU, and GPU requirements for
encompassed computation. Specifying these in the code results in less time for experiment
reproduction; it can also be immediately seen whether one has enough resources to run it
in the first place. These bounds vary greatly for algorithms and models utilized in this
thesis, from a few GB of RAM to almost 400 GB for InLoc processing the InLoc dataset, zero
to eight GPUs for NRIW training, and typically a few CPU cores.\\

Alongside these shell scripts, an attempt to have a reproducible experimental pipeline was
made using \emph{Docker}/\emph{Singularity} and \emph{DVC}.

Docker\footnote{\url{https://www.docker.com}} is an industry-grade platform allowing to
build, test, and deploy applications quickly and robustly. Docker is an example of
container-based virtualization, where a container is a running \uv{image} that packs
everything the software needs for running, including libraries, system tools, code, and
runtime. This approach is suitable for computational cluster environments as the code can
be executed without relying on cluster administrators to install necessary packages
globally for all users, which often leads to software version collisions.
Containerization is a more lightweight virtualization technique compared to classical
virtual machines resulting in quick startup times, lower memory requirements overhead, and
a more user-friendly working experience suitable for both development and
productionalisation. Technically, this is enabled by sharing underlying OS kernel by all
running containers, contrary to virtual machines that are ran on bare metal with the
so-called hypervisor, emulating their own OS kernels separated from other virtual machines
running on the same hardware. To be more specific, the thesis relies on GPU-based
computations; to be able to run GPU workloads in a container, there is an exception to the
mentioned advantage of container-based virtualization to avoiding cluster administrators
globally changing the cluster---suitable GPU drivers must be installed. Especially for all
the renderers described to be able to use off-screen headless rendering (rendering to
textures and saving them without displaying them on a monitor) on NVIDIA GPU cards used by
the CIIRC computational cluster, a driver with bug-less EGL\footnote{EGL is an interface
between Khronos rendering APIs such as OpenGL ES or OpenVG and the underlying native
platform window system, \url{https://www.khronos.org/egl}.} support must be used, which is
something not every driver version satisfies. CUDA and OpenGL libraries are then owned by
each container, communicating with the shared driver on the OS level.

Singularity\footnote{\url{https://sylabs.io}} is a containerization platform similar to
Docker, with one notable exception leading to the adoption of the tool by computational
cluster administrators (including CIIRC's) instead of the otherwise industry-leading and
widely used Docker---it does not require administrative privileges from its users. For
this thesis, descriptions of Docker images to be built are written where applicable. Once
built, images are transformed into Singularity variants runnable on the cluster. This
functionality is supported natively by \verb|singularity| binary as it is a common
use-case for the tool.

Data Version Control (DVC\footnote{\url{https://dvc.org}}) should help traceable and
reproducible science by leveraging the Git version system to also version data,
intermediate results, tie them with the exact code that produced them and thus track all
ideas and experiments. It also can manage workflows which is valuable for defining
reproducible data pipelines. The advantages of this tool are simplicity as it uses
Git---which is a standard tool in code development---and workflow management is done
through simple shell scripting that is well suited for the Slurm environment with running
various Singularity containers as scheduled jobs. Though the idea is promising in the
recent growth of Machine Learning Operations (MLOps), the tool proved unsuitable when used
for datasets consisting of an enormous number of smaller files which is often the case in
computer vision. DVC uses hashes to check consistency and the necessity to recompute some
steps in a workflow, so it may take many hours to run even elementary transformations.
The overhead of these hash computations is considerable, leading to the decision not to
use DVC after all.

\subsection{InLoc localization pipeline}

The implementation of the
pipeline\footnote{\url{https://github.com/Auratons/inlocciirc_demo}} is based on the
author's Matlab sources\footnotei{.}{\url{https://github.com/HajimeTaira/InLoc_demo}} The
source code is unified for better readability and verifiability; it is also generalized,
as the original implementation targets specifically the InLoc
dataset\footnotei{,}{\url{https://github.com/HajimeTaira/InLoc_dataset}}. Furthermore, as
the original code lacks computation of scores and evaluation for a general dataset, the
proposed approach of Pavel Lucivnak for the former is leveraged and further
developed\footnotei{,}{\url{https://github.com/lucivpav/InLocCIIRC_demo}}
\footnote{\url{https://github.com/lucivpav/InLocCIIRC_dataset}}

The outline of the implementation is depicted by~\cref{fig:inloc_pathway}.  Cutouts for a
dataset is a folder structure containing 3~files per DB image---pose file, the image
itself, and a so-called \uv{XYZcut}. The cut is a $\text{M}\times\text{N}\times3$ array
with XYZ coordinates of a surface that would be hit first by a ray cast from the center of
the camera that took the DB image of size $\text{M}\times\text{N}$ (ignoring color
dimension) through given pixel.  Computation of the cuts is not part of the source InLoc
pipeline implementation, so one method of getting the cut from a particular
renderer-generated  depth map and known camera parameters is
implemented\footnotei{.}{\url{https://github.com/Auratons/inlocciirc_dataset}}

A database and query image similarity score used later for image retrieval is computed as
cosine similarity of normalized feature vectors. The original implementation using the
matrix multiplication of query and database features stacked onto each other has immense
memory requirements depleting all resources when executed on the considerably extensive
InLoc dataset, thus some allowed linear algebra adjustments are made, lowering the
requirements to reasonable numbers.

For the image retrieval step, we use 100~closest database images to every given query
photo based on the similarity score, which is the same number of candidates as the origin
article.  For all these candidate poses, after transforming them into formats expected by
the renderers explored in the thesis, we produce candidate renders and use those in the
standard pose verification process described in~\cref{sec:inloc} based on the RootSIFT
descriptors. After reranking the candidate positions based on the image-render similarity,
10~best sorted candidates are outputted as in the original article.

Evaluation is done by comparing angular and spatial L2 distances between the candidate and
the query's true pose, if known. Specifically, for the InLoc dataset, the ground truth
poses for the query set are not publicly disclosed. Only an online evaluation tool
\url{https://www.visuallocalization.net/submission/} returning the fraction of correctly
localized queries within the distance and angular threshold can be used.

\begin{figure}
    \centering
    \begin{tikzpicture}[
    >={Triangle[scale width=0.8]}, thick, black!70,
    text=black,
    every new ->/.style={shorten >=-1pt}
]
    \matrix[row sep=10mm, column sep=8mm] {
        % Zeroth row
        \node (A) [data, fill=darkgreen!20, align=left] {DB Cutouts\\Query Images}; & & \\
        % First row
        \node (B) [process] {File Lists}; &
        \node (C) [data]    {DB\,\&\,Query Lists}; &
        \node (D) [process] {Compute Features}; \\
        % Second row
        \node (G) [data]    {DB$\times$Query Scores}; &
        \node (F) [process] {Compute Scores}; &
        \node (E) [data]    {DB\,\&\,Query Features}; \\
        % Third row
        \node (H) [process] {Image Retrieval}; &
        \node (I) [data]    {Retrieval Candidates}; &
        \node (J) [process] {Pose Estimation}; \\
        % Fourth row
        \node (AA) [data, fill=darkgreen!20] {Point Clouds}; &
        \node (L) [process, draw]            {Renderer}; & 
        \node (K) [data]                     {Candidate Poses}; \\
        % Fifth row
        & \node (M) [data]  {Candidate Renders}; &
        \node (N) [process] {Pose Verification}; \\
        % Sixth row
        \node (P) [process]  {Evaluation}; & &
        \node (O) [data, draw] {Best Query Poses}; \\
    };
    \graph [use existing nodes] {
        A -> B -> C -> D -> E -> F -> G -> H -> I -> J -> K ->["reformatted"'{font=\tiny,inner sep=2pt}] L -> M -> N -> O -> P;
        E -> J; AA -> L; K -> N;
    };
\end{tikzpicture}

    \caption{InLoc algorithm. The implementation
    outline uses terminology from the article. Rectangles represent file(s) on the disk,
    dark green ones denote algorithm inputs, and the rest are intermediate outputs except
    algorithm outputs with a border drawn. Blue nodes of the outline denote processing
    steps.  \uv{DB cutouts} are a database (DB) format the implementation expects.
    \uv{File Lists} step scans the database and query, storing valid found examples and
    query images into a file for further reference. The \uv{Renderer} step highlighted
    with border is the main concern of the thesis.}
    \label{fig:inloc_pathway}
\end{figure}

\subsection{Neural Rerendering in the Wild}

For the Neural Rerendering in the Wild, the original implementation is also
used\footnote{\url{https://github.com/google/neural_rerendering_in_the_wild}} without any
substantial changes, just minor technical enhancements, such as support for alpha channel
processing, etc\footnotei{.}{\url{https://github.com/Auratons/neural_rendering/}}

All scripts needed on the path from a raw dataset to \uv{Aligned Dataset} and \uv{Cutouts}
in~\cref{fig:artwin_dset_pathway}, \cref{fig:imc_dset_pathway}
and~\cref{fig:inloc_dset_pathway} are also implemented in the repository. Aligned dataset
is expected as input to the NRIW training process after packing into a TFRecord, similar
to Cutouts being expected by the InLoc pipeline.

For ARTwin, spherical photos need to be unrolled to 2D with the exact sampling approach as
for the InLoc dataset, resulting in the set of reference images.  To be able to reuse
scripting written initially for the IMC raw dataset, the creation of COLMAP-like camera
and image information structure is implemented alongside unrolling in preprocess script.
For depth information used within the aligned dataset, the point cloud is rendered via the
load data script for ARTwin and IMC data and the render InLoc DB script for the remaining
dataset. These scripts utilize the Pyrender python package internally, transversely
\verb|GL_POINTS| OpenGL primitive for rendering. The approach is a common baseline with
previous works on the topic.  To generate an aligned dataset with point cloud renders
provided by other renderers, the \uv{Generate matrices} step is used, for both splatting
and marching, as they share technically the same headless rendering component mentioned
below.

The aligned dataset is a structure containing, in the simplest case, a triplet of an
image, a color render of the underlying scene representation by a renderer, and the
respective depth map. The triplet forms a deep buffer mentioned in the article.  In the
original article, the authors also use semantic masking in their deep buffers.  However,
when rendering a novel view not seen in the training data, the semantic mask cannot be
obtained from an actual photo. Authors thus train a separate segmentation network between
the partial deep frame buffers and the semantic masks Si to tackle this issue. However,
this makes the network more complex and lowers the prediction time performance.
Semantically segmenting the point cloud might be used, but following~\citet{Bastien}, the
additional complexity is unwanted.

\begin{figure}
    \centering
    \begin{tikzpicture}[
    >={Triangle[scale width=0.8]}, thick, black!70,
    text=black,
    every new ->/.style={shorten >=-1pt},
]
    \matrix (m) [row sep=5mm, column sep=10mm] {
        % First row
        \node (A)       [data, text width=2.1cm, fill=darkgreen!20] {360\textdegree{} Scans\\ Poses}; & [-21mm]
        \node (B)       [process] {preprocess.m}; & [-8mm]
        \node (C)       [data]    {COLMAP Data \\ -- images \\ -- cameras}; & [-3mm]
        \node (helper1) [coordinate] {}; & [-2mm]
        \node (load)    [process] {load\textunderscore{}data.py}; \\ [-6mm]
        % Second row
        \node (F) [data, text width=2.1cm, fill=darkgreen!20] {Point Cloud}; & & & & \\
        % Third row
        & \node (poses) [data] {Camera Poses}; &
        \node (matrices) [process,align=center] {Generate matrices\\{\scriptsize ARTwin variant}}; & &
        \node (hall) [data]    {Aligned Hall}; \\
        % Fourth row
        & \node (renderers) [process] {Dockerized Renderers}; & & & \\ [2mm]
        % Fifth row
        \node (G) [process, anchor=text, xshift=-10mm] {build\textunderscore{}cutouts\textunderscore{}artwin.py}; & &
        \node (E) [data, draw] {Aligned Dataset}; & &
        \node (join) [process] {join\textunderscore{}datasets.py}; \\
        % Sixth row
        \node (H) [data, draw] {Cutouts}; & & & & \\
    };
    \graph [use existing nodes] {
        A -> B -> C -- helper1 -> load -> hall -> join -> E;
        hall -> [dashed] matrices -> [dashed] poses -> [dashed] renderers;
        E -> G;
    };
    \gettikzxy{(F.south west)}{\fx}{\fy}
    \gettikzxy{(renderers.south)}{\hx}{\hy}
    \gettikzxy{(F.south)}{\fsx}{\fsy}
    \gettikzxy{(G.north)}{\gx}{\gy}spot
    \gettikzxy{(G.south)}{\gsx}{\gsy}
    \gettikzxy{(H.north)}{\hhx}{\hhy}
    \draw [very thin, densely dash dot, black] ($ (\fx, \hy) + (-0.1, -0.15) $) rectangle ($ (load.north east) + (0.1, 0.51) $);
    \draw [very thin, densely dash dot, black] ($ (\fx, \hy) + (-0.15, -0.2) $) rectangle ($ (load.north east) + (0.05, 0.46) $);
    \draw [rounded corners] (F) -| (helper1) -- (load.west);
    \draw [->] (F) -- (\fsx, \gy);
    \draw [->] (\fsx, \gsy) -- (\fsx, \hhy);
    \draw [rounded corners, ->, dashed] (renderers) -| ($ (hall.south west) + (5mm, 0)$);
    \draw [rounded corners, ->, dashed] ($ (F.south) + (1mm, 0) $) |- (renderers);
    \node [above=of load.north, yshift=-5.5mm, font=\tiny, text=black] {repeated for 2 halls};
\end{tikzpicture}

    \caption{ARTwin dataset pathway. The
    schema display transformations the ARTwin dataset undergoes in order to get either
    \uv{Aligned Dataset} expected by the Neural Rerendering in the Wild DNN training or
    \uv{Cutouts} for the localization pipeline. Rectangles represent file(s) on the disk,
    dark green ones denote algorithm inputs, and the rest are intermediate outputs except
    algorithm outputs with a border drawn. Blue nodes of the outline denote processing
    steps. The dashed paths are used to incorporate additional steps needed for
    non-default renderers.  The default rendering with Pyrender is implemented in the load
    data script.}
    \label{fig:artwin_dset_pathway}
\end{figure}

\begin{figure}
    \centering
    \begin{tikzpicture}[
    >={Triangle[scale width=0.8]}, thick, black!70,
    text=black,
    every new ->/.style={shorten >=-1pt}
]
    \matrix[row sep=5mm, column sep=10mm] {
        % First row
        & [-5mm]
        \node (A) [data, text width=2.1cm, fill=darkgreen!20] {3D Scans\\ Poses \\ Cutouts}; & [-5mm]
        \node (B) [process] {render\textunderscore{}inloc\textunderscore{}db.py}; & \\ [-4.5mm]
        % 1.5 row
        & \node (AA) [data, text width=2.1cm, fill=darkgreen!20] {Point Cloud}; & & & & \\
        % Second row
        \node [coordinate] (helper) {}; & \node (C) [data] {Camera Poses}; &
        \node (D) [process,align=center] {Generate matrices\\{\scriptsize InLoc variant}}; &
        \node (E) [data]    {Aligned Buildings}; \\
        % Third row
        & \node (CC) [process] {Dockerized Renderers}; & & \\
        % Fourth row
        & & \node (F) [data, draw] {Aligned Dataset}; & 
        \node (G) [process] {export\textunderscore{}to\textunderscore{}nriw.py}; \\
        % Fifth row
        & \node (J) [process] {build\textunderscore{}cutouts\textunderscore{}inloc.py}; &
        \node (I) [data, draw] {Cutouts}; & \\
    };
    \graph [use existing nodes] {
        A -> B;
        E -> G -> F;
        E -> [dashed] D -> [dashed] C -> [dashed] CC;
        J -> I;
    };
    \draw [rounded corners, ->] (B) -| (E);
    \draw [rounded corners, ->, dashed] (CC) -| ($ (E.south west) + (7mm, 0)$);
    \draw [rounded corners, ->] (A) -| (helper) |- (J);
    \draw [rounded corners, ->] (AA) -| (B);
    \draw [rounded corners, ->, dashed] (AA) -| ($ (helper) + (1mm, 0)$) |- (CC);
\end{tikzpicture}

    \caption{InLoc Dataset pathway. The
    schema display transformations the InLoc dataset undergoes in order to get either
    \uv{Aligned Dataset} expected by the Neural Rerendering in the Wild DNN training or
    \uv{Cutouts} for the localization pipeline. Rectangles represent file(s) on the disk,
    dark green ones denote algorithm inputs, and the rest are intermediate outputs except
    algorithm outputs with a border drawn. Blue nodes of the outline denote processing
    steps. The dashed paths are used to incorporate additional steps needed for
    non-default renderers.  The default rendering with Pyrender is implemented in the
    render InLoc db script.}
    \label{fig:inloc_dset_pathway}
\end{figure}

\begin{figure}
    \centering
    \begin{tikzpicture}[
    >={Triangle[scale width=0.8]}, thick, black!70,
    text=black,
    every new ->/.style={shorten >=-1pt}
]
    \matrix[row sep=5mm, column sep=10mm] {
        % First row
        & [-4mm] \node (A) [data, fill=darkgreen!20] {COLMAP Data\\ -- images\\ -- cameras}; &
        \node (B) [process]    {load\textunderscore{}data.py}; & \\
        % Second row
        \node [coordinate] (helper) {}; &
        \node (D) [process,align=center] {Generate matrices\\{\scriptsize IMC variant}}; &
        \node (AA) [data, fill=darkgreen!20] {Point Cloud}; & \\
        % Third row
        & \node (CC) [data] {Camera Poses}; &
        \node (E) [process] {Dockerized Renderers}; &
        \node (F) [data, draw]    {Aligned Dataset};\\
        % Fourth row
        & \node (G) [process] {build\textunderscore{}cutouts\textunderscore{}colmap.py}; &
        \node (H) [data, draw] {Cutouts}; & \\
    };
    \graph [use existing nodes] {
        A -> B;
        A -> [dashed] D -> [dashed] CC -> [dashed] E -> [dashed] F;
        G -> H;
        AA -> [dashed] E; AA -> B;
    };
    \draw [rounded corners, ->] (B) -| (F);
    \draw [rounded corners, ->] (A) -| (helper) |- (G);
\end{tikzpicture}

    \caption{IMC Dataset pathway. The schema
    display transformations the IMC dataset undergoes in order to get either \uv{Aligned
    Dataset} expected by the Neural Rerendering in the Wild DNN training or \uv{Cutouts}
    for the localization pipeline. Rectangles represent file(s) on the disk, dark green
    ones denote algorithm inputs, and the rest are intermediate outputs except algorithm
    outputs with a border drawn. Blue nodes of the outline denote processing steps. The
    dashed paths are used to incorporate additional steps needed for non-default
    renderers.  The default rendering with Pyrender is implemented in the load data
    script.}
    \label{fig:imc_dset_pathway}
\end{figure}

The model is then trained with the same staged approach, where the appearance encoder is
first pretrained on a proxy task with the triplet loss. As in the original article,
$256\times256$ central crops of the deep buffers are used. For all the datasets, the whole
train/val sets are used as the model is scene-dependent, which is especially true for the
InLoc dataset, where the test set covers only a portion of the database.

Finally, the training parameters are also used following the article, only with scaled
batch sizes according to GPU memory available in the CIIRC cluster DGX nodes.

\subsection{Spherical Ray Marcher}

The spherical ray marcher is, on the high level, an
OpenGL\footnote{\url{https://www.opengl.org}} application with both interactive and
headless rendering capabilities. Interactive rendering includes FPS camera moved by
keyboard, displaying real-time point cloud on user's monitor.  Headless mode generates
specific view renders on the fly, without a monitor, directly to files on the disk, and it
serves for dataset generation.

The final texture depicting the requested view is computed from the main ray-casting loop
implemented in CUDA\footnote{\url{https://developer.nvidia.com/cuda-toolkit}} due to usage
of a KD-tree implementation based on the
FLANN\footnote{\url{https://github.com/flann-lib/flann}} project. As the NRIW training and
cutout computations require depth maps, the implementation also provides outputting depth
texture alongside any RGB render.

The algorithm needs radii for points to be rendered, the
implementation\footnote{\url{https://github.com/Auratons/renderer_ray_marching}} can
compute those and cache them alongside the input point cloud.  Requested renders are
expected to be specified by the respective camera poses.


\subsection{Surface Splatting}

For surface splatting, Sebastian Lipponer's implementation was
used\footnote{\url{https://github.com/sebastianlipponer/surface_splatting}} as a base, the
project\footnote{\url{https://github.com/Auratons/renderer_surface_splatting/}} was
enhanced with the same headless rendering capability as in the case of the ray marching.
Also, a mechanism for loading radii of the point cloud being rendered was added, also
reusing the component from the spherical ray marcher. Furthermore, from the same reason of
the NRIW training process, depth buffer content is made accessible as another output from
the renderer, the original implementation produced only RGB outputs.  Finally, a bug in
camera handling of the underlying interactive rendering library GLviz of the same author
was identified and resolved\footnote{\url{https://github.com/Auratons/glviz}} to have
rendered views for the same view matrices unified across all renderers used. The bug was
not obvious when using the FPS camera for simple exploration of the model being displayed,
but when compared to outputs of a proper computer-vision grade renderer it became obvious.


\begin{table}
\caption{Comparison of input format expectations of algorithms used in the
thesis from the transformations perspective. The upper row presents the algorithms, and
the lower one contains abbreviations of conventions, where \emph{RC} means rendering
convention, \emph{CP} means camera pose matrix. InLoc pipeline is agnostic to matrix
convention as far as it is consistent with data generation. NRIW itself is trained only
with images, noteworthy those are generated by the preceding rendering approaches with
their conventions.}
\centering
    \begin{tabular}{c c c c c}
    \toprule
    Marching & Splatting & Pyrender & InLoc Pipeline & NRIW\\
    \midrule
    CP, RC & CP, RC & CP, RC & Agnostic & -- \\
    \bottomrule
    \end{tabular}
\label{tab:agorithm_conventions}
\end{table}
