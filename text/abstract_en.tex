%%% A template for a simple PDF/A file like a stand-alone abstract of the thesis.

\documentclass[12pt]{report}

\usepackage[a4paper, hmargin=1in, vmargin=1in]{geometry}
\usepackage[a-2u]{pdfx}
\usepackage[utf8]{inputenc}
\usepackage[T1]{fontenc}
\usepackage{lmodern}
\usepackage{textcomp}

\begin{document}

Visual localization is the problem of estimating the 6~degrees of freedom
camera pose from which a query image was taken relative to a known
reference scene representation. It is the key for applications such as
Augmented, Mixed, and Virtual Reality, as well as autonomous robotics
such as drones or self-driving cars.

This thesis focuses on a visual localization pipeline, especially on
its pose verification and reranking step. The pipeline uses 3D point clouds
and 2D-3D correspondences between the query image and 3D scene points for
candidate camera poses estimations. The thesis explores point cloud
rendering approaches as they are utilized in the pipeline and the
verification step---the render of the discretized scene from a given
candidate position is compared to the actual query image to asses if the
given couple depicts the same place.

One of the main challenges of such rendering is occlusion handling. Due
to the sparsity of points employed for otherwise continuous real world
representation, information about what lies in the front and what is
hidden can be easily lost when projected to the 2D image. Rendering
approaches explored in this thesis focus on the challenge directly or
as a component of a novel view synthesis DNN-based renderer.
Rendering influence on localization performance is investigated.

\end{document}
