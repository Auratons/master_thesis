%%% A template for a simple PDF/A file like a stand-alone abstract of the thesis.

\documentclass[12pt]{report}

\usepackage[a4paper, hmargin=1in, vmargin=1in]{geometry}
\usepackage[a-2u]{pdfx}
\usepackage[utf8]{inputenc}
\usepackage[T1]{fontenc}
\usepackage{lmodern}
\usepackage{textcomp}

\providecommand{\uv}[1]{\raisebox{-1.3ex}{’’}#1‘‘}

\begin{document}

Vizuální lokalizace je problém odhadování parametrů šesti stupňů volnosti
pozice kamery, z~níž byla pořízena dotazovaná fotografie, přičemž pozice
je vztažena ke známé reprezentaci referenčního prostředí. Řešení tohoto
problému je klíčové v~aplikacích jako jsou rozšířená, smíšená a~virtuální
realita, stejně tak v~oblasti autonomní robotiky zahrnující drony
a~samořiditelné automobily.

Tato práce se soustředí na vizuální lokalizační algoritmus, zejména na
jeho verifikační a~přeřazovací krok. Tento algoritmus interně využívá
třídimenzionální mračna bodů a~hledání korespondencí mezi těmito body
a~dotazovanou fotografií pro nalezení odhadů kandidátních pozic kamery.
Práce zkoumá přístupy k~renderování mračen bodů a jejich využití v~rámci
algoritmu a jeho verifikačního kroku~-- render diskretizovaného prostředí
z~konkrétní kandidátní pozice se v něm porovnává s~danou dotazovanou fotografií
za účelem určení toho, zda oba pohledy zobrazují to samé místo.

Jedna z~hlavních výzev renderingu diskretizovaného prostředí jsou okluze.
Kvůli řídkosti bodů využitých jako reprezentace jinak spojitého reálného
světa může být informace o~tom, co leží v~popředí a co v~pozadí, lehce
ztracena při promítnutí bodů na dvoudimenzionální obraz. Přístupy
k~renderování zkoumané v~této práci se soustředí na renderování bodů přímo
nebo jako komponentu rendereru \uv{nových pohledů} využívající hlubokých
neuronových sítí. Je zde prověřen vliv těchto renderovacích přístupů na
přesnost lokalizace.

\end{document}
